\documentclass{beamer}
\usepackage{xeCJK}
\usetheme{Copenhagen}
\usecolortheme{beaver}
\usepackage{float}
\usepackage{graphicx}
\usepackage[boxed]{algorithm2e}

\title{Skip List}
\author[Kvrmnks]{Kvrmnks \\ tayyx2000@163.com}
\date{2020.11.11}
\begin{document}
	\begin{frame}
		\titlepage
	\end{frame}
	\section{简介}
	\subsection{Skip List是什么}

	\begin{frame}
		\frametitle{Skip List是什么}
		\begin{figure}[H]
			\centering
			\includegraphics[scale=0.25]{./img/what_skip_list_is.jpg}
			
		\end{figure}
		是链表,高度不一样,指针多了不少,是有序的,指针有的是跳着指的。
	\end{frame}

	\begin{frame}
		\frametitle{Skip List的功能}
		\begin{enumerate}
			\item 查找 \quad 期望$O(\log n)$,最坏$O(n)$
			\item 插入 \quad 期望$O(\log n)$,最坏$O(n)$
			\item 删除 \quad 期望$O(\log n)$,最坏$O(n)$
			\item 前驱 \quad 期望$O(\log n)$,最坏$O(n)$
			\item 后继 \quad 期望$O(\log n)$,最坏$O(n)$
			\item 第$k$大 \quad 期望$O(\log n)$,最坏$O(n)$
			\item $rank$ \quad 期望$O(\log n)$,最坏$O(n)$
			\item 空间复杂度 \quad 期望$O(n)$, 最坏$O(n\log n)$
		\end{enumerate}
	\end{frame}
	
	\subsection{Skip List有哪些优势}
	\begin{frame}
		\frametitle{Skip List有哪些优势}
		\begin{enumerate}
			\item 实现简单
			\item 期望下有和一般的平衡树一样的复杂度
			\item 更好地支持并行
			\item finger search
		\end{enumerate}
	\end{frame}
	\subsection{Skip List是怎么实现的}
	\begin{frame}
		\frametitle{Skip List是怎么实现的}
		\begin{figure}[H]
			\centering
			\includegraphics[scale=0.25]{./img/what_skip_list_is.jpg}
		\end{figure}
		每个元素都有一个高度\\
		每个高度都有一个指针\\
		每个指针向右指向第一阻挡到它的地方\\
	\end{frame}

	\section{基操}
	
	\subsection{插入}
	\begin{frame}
		\frametitle{插入}
		\begin{figure}[H]
			\includegraphics[scale=0.22]{./img/find.jpg}
		\end{figure}
		从起点的最高点出发,能往右走就往右走,不行就降低高度。\\
		维护查找过程中每层最后一个, 为了维护每个指针。\\
		找到要插入元素的位置之后,直接放进去,更新指针。\\
		不需要别的操作保持平衡!
	\end{frame}

	\begin{frame}
		\frametitle{每个元素的高度如何确定}
		一个元素的高度是随机确定的,具体随机方式如下\\
		有一个常数$p$, $0 \leq p \leq 1$, 这个元素高度为$1$的概率是$p$, 为$2$的概率是$p^2$ $\cdots$ 为$m$的概率为$p^m$ \\
		其中特别规定最大高度为$\log{n}$
	\end{frame}

	\subsection{查找}
	\begin{frame}
		\frametitle{查找}
		\begin{figure}[H]
			\includegraphics[scale=0.25]{./img/insert.jpg}
		\end{figure}
		从起点的最高点出发,能往右走就往右走,不行就降低高度。
	\end{frame}

	\subsection{删除}
	\begin{frame}
		\frametitle{删除}
		\begin{figure}[H]
			\includegraphics[scale=0.25]{./img/find.jpg}
		\end{figure}
		依旧要维护查找时每层最后一个元素,来维护指针。\\
		不需要别的操作保持平衡!
	\end{frame}

	\subsection{rank}


	\begin{frame}
		\frametitle{求第$k$大}
		此处应有一张rank tree的图。
	\end{frame}

	\begin{frame}
		\frametitle{求第$k$大}
		给每条边加权
		\begin{figure}[H]
			\includegraphics[scale=0.27]{./img/rank.jpg}
		\end{figure}
		边权代表走了这条边之后排名增加了多少
	\end{frame}

	\begin{frame}
		\frametitle{求rank}
		我可以二分呀(X)\\
		此处又应有一张rank tree的图。
	\end{frame}

	\begin{frame}
		\frametitle{求rank}
		同上面的方法一样维护边权,查找过程中累加每条边的长度
		\begin{figure}[H]
			\includegraphics[scale=0.27]{./img/rank.jpg}
		\end{figure}
	\end{frame}

	\begin{frame}
		\frametitle{求前驱和后继}
		这...
	\end{frame}

	
	\section{好玩的东西}
	\subsection{来想想怎么改进链表}
	\begin{frame}
		\frametitle{怎样让链表快一点}
		\begin{figure}[H]
			\includegraphics[scale=0.3]{./img/faster.jpg}
		\end{figure}
		\pause
		分块! \\
		\pause
		\begin{enumerate}
			\item<3-> 怎么分最好?\uncover<5->{$\sqrt{n}$分块}
			\item<4-> 支持删除插入吗?\uncover<6->{同样是$O(\sqrt{n})$的复杂度}
	\end{enumerate}
	\end{frame}
	\begin{frame}
		\frametitle{块状链表}
		假设一共有$n$个数,假设将链表分成$\frac{n}{a}$块,块的大小是$a$

		于是每次查找的复杂度是$O(\frac{n}{a}+a)$,由均值不等式$a = \sqrt{n}$时,复杂度达到最小,这时复杂度是$O(\sqrt{n})$
	\end{frame}

	\begin{frame}
		\frametitle{块状链表}
		考虑怎样维护插入和删除。\\
		\begin{figure}[H]
			\includegraphics[scale=0.2]{./img/block.jpg}
		\end{figure}
		先规定每块的大小限制再考虑块数\\
		如果插入的块比$\sqrt{n}$,把块分裂成两个\\
		删除直接在所在块中删除 \\
		将新相接的块尝试合并 \\
	\end{frame}

	\begin{frame}
		\frametitle{块状链表}
		假设第$i$块的大小为$s_{i}$,
		根据上面的规则有$$s_{i}+s_{i-1}\geq \sqrt{n}$$
		累加这个不等式,可以得到$$s_{1}+2*\sum_{i=2}^{m-1}s_{i}+s_{m} \geq m\sqrt{n}$$
		放缩一下得到
		$$2n\geq m\sqrt{n}$$$$m\leq 2\sqrt{n}$$
		于是总复杂度为$$O(\sqrt{n})$$
	\end{frame}
	\begin{frame}
		\frametitle{再快一点?}
		\begin{figure}[H]
			\includegraphics[scale=0.25]{./img/fasterter.jpg}
		\end{figure}
		\begin{enumerate}
			\item 查询是$O(\log n)$了! \quad \pause 可以证明每层横着走最多一次
			\pause
			\item 删除和加入都很困难
		\end{enumerate}
	\end{frame}

	\subsection{无内鬼,来点复杂度知识}
	\begin{frame}
		\frametitle{各种复杂度}
		\begin{enumerate}
			\item 什么是期望复杂度
			\item 什么是最好复杂度
			\item 什么是最坏复杂度
			\item 什么是均摊复杂度
		\end{enumerate}
	\end{frame}

	\begin{frame}
		\frametitle{最好和最坏复杂度}
		\begin{enumerate}
			\item 什么是最好复杂度
			\item 什么是最坏复杂度
		\end{enumerate}
		冒泡排序
	\end{frame}

	\begin{frame}
		\frametitle{期望复杂度和均摊复杂度}
		\begin{enumerate}
			\item 什么是期望复杂度
			\item 什么是均摊复杂度
		\end{enumerate}
	\end{frame}

	\begin{frame}
		\frametitle{期望复杂度和均摊复杂度}
		为什么要关心最坏复杂度和最好复杂度?

		参数化算法 \quad 缝合怪
	\end{frame}

	\begin{frame}
		\frametitle{$\Omega$ O $\Theta$的区别...}
		\begin{figure}[H]
			\centering
			\includegraphics[scale=0.27]{./img/Ocomplexity.jpg}
		\end{figure}
		\pause
		其实这个跟最好最坏情况没什么关系...
	\end{frame}

	\begin{frame}
		\frametitle{时间复杂度}
		可以证明查找的复杂度在期望情况下是$\Theta(\log{n})$ \\
		但这并没有什么卵用,大家其实并不在乎$\Theta$和O
	\end{frame}

	\subsection{高级操作}
	\begin{frame}
		\frametitle{区间求和}
		\begin{figure}[H]
			\centering
			\includegraphics[scale=0.25]{./img/rank.jpg}
		\end{figure}
		求rank实际上就是一种弱的区间求和,只需要再额外维护一个边权表示走了这条边,走过的元素的权值和增加了多少
	\end{frame}

	\begin{frame}
		\frametitle{它是个链表}
		区间移动,文本编辑器。
	\end{frame}

	\begin{frame}
		\frametitle{可以用来实现ETT}
		\begin{enumerate}
			\item 欧拉序
			\item 动态树
			\item link-cut
			\item 维护子树信息
			\item 维护到根信息
		\end{enumerate}
	\end{frame}

	\begin{frame}
		\frametitle{确定性skip list}
		似乎能和2-3树对应...?
	\end{frame}

	
	\section{空间复杂度}
	\begin{frame}
		\frametitle{空间复杂度}
		$$E[X] = \sum_{i=1}^{n}E[X_{i}] = \sum_{i=1}^{n}\sum_{j=1}^{\infty}jp^{j} = \sum_{i=1}^{n}\frac{p}{(p-1)^2} = \frac{np}{(p-1)^2}$$
		取$p=\frac{1}{2}$,$E[X] = 2n$,期望空间复杂度$O(n)$ \\
		显然最坏情况下空间复杂度为$O(n\log n)$
	\end{frame}
	\section{时间复杂度}
	\begin{frame}
		\frametitle{先感性理解一下}
		\begin{figure}[H]
			\includegraphics[scale=0.25]{./img/fasterter.jpg}
		\end{figure}
	\end{frame}
	\begin{frame}
		\frametitle{查找的复杂度}
		考虑反着思考复杂度
		\begin{enumerate}
			\item 跳到最高处的步数
			\item 跳到起点的步数
		\end{enumerate}
	\end{frame}
	\begin{frame}
		\frametitle{跳到最高处的步数}
		考虑一个无限长的Skip List
		\begin{figure}[H]
			\centering
			\includegraphics[scale=0.2]{img/complexity.jpg}
		\end{figure}
		\pause
		$$E[Y] = pE[Y]+(1-p)E[Y-1]+1$$
	\end{frame}
	\begin{frame}
		\frametitle{跳到最高处的步数}
		$$E[Y] = (1-p)E[Y]+pE[Y-1]+1$$
		\pause
		\\化简得\\
		$$E[Y] - E[Y-1] = \frac{1}{p}$$\\
		\pause
		$$E[0] = 0$$
		\pause
		于是$E[Y] = \frac{Y}{p}$
		\pause
		当$Y = [\log{n}]-1, p=\frac{1}{2}$,有$E[Y] = 2[\log{n}]-2$
	\end{frame}
	\begin{frame}
		\frametitle{跳到起点的步数}
		分析最高层有多少个数。
		$$E[Z] = \sum_{i=1}^{n}\sum_{j=[\log n]}^{\infty}jp^{j} 
		= \frac{np^{[\log{n}]} (p+[\log{n}]-p[\log{n}])}{(p-1)^2} $$
		\pause
		取$p=\frac{1}{2}$\\
		$$E[Z] \leq 2[\log{n}]+2$$
	\end{frame}
	\begin{frame}
		\frametitle{时间复杂度}
		\begin{enumerate}
			\item 跳到最高处的步数 \quad $E[Y] = 2[\log{n}]-2$
			\item 跳到起点的步数 \quad $E[Z] \leq 2[\log{n}]+2$
		\end{enumerate}
		期望复杂度为$O(\log{n})$ \\
	\end{frame}

	\begin{frame}
		\frametitle{时间复杂度}
		\begin{enumerate}
			\item 查找 \quad 期望$O(\log n)$,最坏$O(n)$
			\item 插入 \quad 期望$O(\log n)$,最坏$O(n)$
			\item 删除 \quad 期望$O(\log n)$,最坏$O(n)$
			\item 前驱 \quad 期望$O(\log n)$,最坏$O(n)$
			\item 后继 \quad 期望$O(\log n)$,最坏$O(n)$
			\item 第$k$大 \quad 期望$O(\log n)$,最坏$O(n)$
			\item $rank$ \quad 期望$O(\log n)$,最坏$O(n)$
		\end{enumerate}
	\end{frame}
	\subsection{其他操作的复杂度}
	\begin{frame}
		\begin{enumerate}
			\item 插入 \quad 查找时维护每个高度最后的一个结点
			\item 删除 \quad 查找时维护每个高度最后的一个结点
			\item 前驱 \quad 本质就是查找
			\item 后继 \quad 查找之后跳到最底层再往前跳
			\item 第$k$大 \quad 本质就是搜索
			\item rank \quad 本质就是搜索
		\end{enumerate}
	\end{frame}
	\section*{}
	\begin{frame}
		感谢倾听!
	\end{frame}
\end{document}